\documentclass{article}
\usepackage[utf8]{inputenc}
\usepackage[russian]{babel}
\usepackage{vmargin}
\setpapersize{A4}
\usepackage{indentfirst}
\usepackage{graphicx}
\usepackage{amsmath}
\usepackage{amsfonts}
\usepackage{pifont}
\usepackage{titlesec}
\DeclareMathOperator{\sgn}{sgn}
\DeclareMathOperator{\conv}{conv}
\DeclareMathOperator{\const}{const}
\DeclareMathOperator{\Argmax}{Argmax}
\DeclareMathOperator{\tr}{tr}
\DeclareMathOperator{\re}{Re}
\clubpenalty=9999
\widowpenalty=9999
\sloppy

% Стилизация

%\titleformat{\chapter}[block]
 % {\normalfont\bfseries\Huge}{\thechapter.}{10pt}{}
%\titleformat{\section}[block]
 % {\normalfont\bfseries\Large}{\thesection \:}{10pt}{}

\begin{document}
% Команды


% Титульный лист

\thispagestyle{empty}
\begin{center}
\vspace{-3cm}
\includegraphics[width=0.5\textwidth]{msu}\\
\begin{small}
{\scshape Московский государственный университет имени М.~В.~Ломоносова}\\
Факультет вычислительной математики и кибернетики\\
Кафедра системного анализа
\end{small}
\vfill
{\Large Отчёт по практикуму}\\
\vspace{1cm}
{\LARGE\bfseries <<Численное решение задачи суперхеджирования с помощью уравнения Беллмана-Айзекса>>}
\end{center}
\vspace{1cm}
\begin{flushright}



\textit{Студент 515 группы}\\
Г.~А.~Садков\\
\vspace{5mm}
\textit{Руководитель практикума}\\
С.~Н.~Смирнов
\end{flushright}
\vfill
\begin{center}
Москва, 2020
\end{center}
\pagebreak

% Оглавление

\tableofcontents
\pagebreak

% Постановка задачи

\section{Постановка задачи}

Необходимо численно решить уравнение Беллмана-Айзекса для задачи гарантированного суперхеджирования опциона американского типа с заданной функцией выплат
для $n$ (рисковых) активов при наличии торговых ограничений. Предполагается, что на
рынке нет транзакционных издержек.
\\
Задача рассматривается в дискретном времени на конечном горизонте $N$ . Пусть
$x_t = (x^1_t , . . . , x^n_t ) \in (0, \infty), \, n$ --- вектор (дисконтированных) цен рисковых активов в
момент $t \in [0, N]$. Будем называть предысторией цен (к текущему моменту времени
$t = 1, . . . , N $) последовательность $\bar{x}_{t-1} = (x_0 , . . . , x_{t-1} )$. Считается, что безрисковый актив имеет постоянную цену, равную единице, и не подвержен торговым ограничениям. Стратегией хеджера называется последовательность векторов $h_1 , . . . , h_N$, описывающая структуру портфеля рисковых активов, такая что $h_t \in D_t (\bar{x}_{t-1} ) \subseteq \mathbb{R}_n$ при априори заданных торговых ограничениях, задаваемых многозначными отображениями $D_t (·)$. \\
Неопределённая динамика цен может быть описана в аддитивной форме: $x_t = x_{t-1} + y_t;$ предполагается, что приращение цен $y_t \in K_t (\bar{x}_{t-1} )$, где компактнозначные отображения $K_t (·)$ имеют смысл заданных априори допустимых сценариев изменения
цен. От чисто детерминистского описания модели можно перейти к описанию динамики цен посредством смешанных стратегий рынка из класса $P_t (\bar{x}_{t-1} )$ вероятностных
мер с носителями, содержащимися в $K_t (\bar{x}_{t-1} )$, который содержит все меры, сосредоточенные в одной точке из $K_t (\bar{x}_{t-1} )$ (меры Дирака). Обозначим $P^t_n (\bar{x}_{t-1} )$ — множество вероятностных мер из $P_t (\bar{x}_{t-1} )$, сосредоточенных не более чем в $n + 1$ точке.
 \\
Пусть $g_t (\bar{x}_t ), t = 0, . . . , N $, --- функции выплат по американскому опциону. Соответствующая целевая функция $v_t^{\star} (\bar{x}_t ), t = 0, . . . , N$ , --- резервы на (гарантированное)
покрытие текущих и будущих обусловленных обязательств по проданному опциону.
Считаем, что условия задачи таковы, что функции $v_t^{\star}$ полунепрерывны сверху. Тогда
имеет место игровое равновесие и уравнения Беллмана-Айзекса, отражающие игровую
интерпретацию для исходной (детерминистской) задачи, могут быть записаны в виде уравнений Беллмана, где задача ценообразования отделена от задачи хеджирования:
\begin{equation}\label{bellman}
	\begin{array}{c}
		v^{\star}_N(\bar{x}_{t-1}) = g_N(\bar{x}_{t-i}),\\
		v^{\star}_{t-1} \left( \bar{x}_{t-1} \right) = g_{t-1}(\bar{x}_{t-1}) \bigvee \underset{ Q \in P_i^n (\bar{x}_{t-1} )}{ \sup } \left[ \int v^{\star}_t \left( \bar{x}_{t-1}, x_{t} + y\right) Q(dy) - \sigma _{D_t(\bar{x}_{t-1}) \left( \int y Q(dy)\right) }\right], \\
		t = 1, \dots, N,		
	\end{array}
\end{equation}
где $\sigma D_t (\bar{x}_{t-1} ) (·)$ – опорная функция множества $D_t (\bar{x}_{t-1})$. Уравнение \ref{bellman} может быть решено для каждого $t$ в два этапа:
\begin{itemize}
	\item для фиксированного $z \in K_t (\bar{x}_{t-1} )$ находится функция	$ut (\bar{x}t-1 , z) =
	sup
	vt\star (\bar{x}t-1 , xt-1 + y)Q(dy).
	n
	Q \in P
	t (\bar{x}t-1 )
	!
	yQ(dy)=z.$
	\item функция цены находится как
	$ \star
	vt-1
	(\bar{x}t-1 ) = gt-1 (\bar{x}t-1 )
	sup
	z\in Kt (\bar{x}t-1 )
	'
	ut (\bar{x}t-1 , z) - \sigma Dt (\bar{x}t-1 ) (z) .
	$
\end{itemize}
Функция $z \leftarrow u_t (\bar{x}_{t-1} ,z)$, определённая посредством (2), совпадает с вогнутой оболочкой1 функции $f (y) =
) v \star (\bar{x} , x + y), y \in K (\bar{x} ),
t-1
t-1
t t-1
t
$ в точке $z$. Таким образом,
для решения задач требуется находить вогнутую оболочку функции.
Для описания неопределённой динамики цен модели будем использовать мультипликативную форму, эквивалентную аддитивной. Обозначим
$(
+
)xit = mit xit-1 , i = 1, . . . , n,
+
*mt = (m1 , . . . , mn ) \in Et (·),
t
t$
(4)
где компактное множество $Et (·)$ задаётся как множество мультипликаторов $m =
(m1 , . . . , mn )$, таких что точка $y = (y 1 , . . . , y n )$ с координатами $y i = (mi - 1)xit-1 $,
$i = 1, . . . , n$ принадлежит $Kt (·)$. Цены рисковых активов в каждый момент времени
$t = 0, . . . , n$ дол<ны быт0 положительными, т.е. $xt = (x1t , . . . , xnt ) \in (0, \infty)n $, что равносильно положительности цен в начальный момент и условию $Et (·) \subseteq (0, \infty)^n$ . При этом
начальные цены предполагаются известными и фиксированными.
Дана система дифференциальных уравнений:

\begin{equation}
\begin{cases}\label{main:syst}
\dot u =\frac{a u^2}{N + u} - gu - buv + d_1 u_{x x}\\
\dot v = - cv + duv + d_2 v_{xx}
\end{cases}
\end{equation}
где $a$. $b$, $c$, $d$, $g$, $N$, $d_1$, $d_2$ --- положительные параметры. Для этой системы требуется:

\begin{enumerate}

\item Сделать замены переменных;

\item Исследовать фазовый портрет нераспределённой системы (неподвижные точки, их тип, изоклины);

\item Проверить наличие волновых решений распределённой системы;

\item Выбрать подходящее решение и скорость волны;

\item Привести иллюстрации и графики фазовых портретов и решений.

\end{enumerate}
\pagebreak

\section{Замена переменных}

Введём следующую замену переменных:
\begin{equation*}
\begin{aligned}
&\tilde{u} = \alpha u,\\
&\tilde{v} = \beta v,\\
&\tilde{t} = \gamma t,\\
&\tilde{x} = \delta x.
\end{aligned}
\end{equation*}
Тогда система (\ref{main:syst}) примет вид:
\begin{equation*}
\begin{cases}
\frac{\gamma}{\alpha} \tilde{u}_{\tilde{t}} =\frac{a \tilde{u}^2}{\alpha^2 \left(N + \frac{\tilde{u}}{\alpha}\right)} - \frac{g \tilde{u}}{\alpha} - \frac{b \tilde{u} \tilde{v}}{\alpha \beta} + \frac{d_1 \delta^2 \tilde{u}_{\tilde{x} \tilde{x}}}{\alpha}\\
\frac{\gamma}{\beta} \tilde{v}_{\tilde{t}} = -\frac{c \tilde{v}}{\beta} + \frac{d \tilde{u} \tilde{v}}{\alpha \beta} + \frac{d_2 \delta^2 \tilde{v}_{\tilde{x} \tilde{x}}}{\beta}
\end{cases}
\end{equation*}

Возьмём коэффициенты из замены равными:
\begin{equation*}
\begin{aligned}
&\alpha = \frac{1}{N},\\
&\beta = \frac{b}{dN},\\
&\gamma = dN,\\
&\delta = \sqrt{\frac{dN}{d_2}}.
\end{aligned}
\end{equation*}
Также положим 
\begin{equation*}
\begin{aligned}
&\tilde{u} = u,\\
&\tilde{v} = v,\\
&\tilde{t} = t,\\
&\tilde{x} = x,
\end{aligned}
\end{equation*}
и получим следующую систему:
\begin{equation*}
\begin{cases}
\dot u =\frac{a u^2}{dN(1 + u)} - \frac{gu}{dN} - uv + \frac{d_1}{d_2} u_{x x}\\
\dot v = - \frac{cv}{dN} + uv + v_{xx}
\end{cases}
\end{equation*}

Теперь введём обозначения:
\begin{equation*}
\begin{aligned}
&A = \frac{a}{dN},\\
&B = \frac{g}{dN},\\
&C = \frac{c}{dN},\\
&D = \frac{d_1}{d_2},
\end{aligned}
\end{equation*}
и получим эквивалентную (\ref{main:syst}) систему, с которой и будем работать в дальнейшем:
\begin{equation}
\begin{cases}\label{trans:syst}	
\dot u =A\frac{u^2}{1 + u} - Bu - uv + D u_{x x}\\
\dot v = - Cv + uv + v_{xx}
\end{cases}
\end{equation}

\section{Исследование нераспределённой системы}

Нераспределённая система в данном случае выглядит следующим образом:
\begin{equation}
\begin{cases}\label{undist:syst}	
\dot u =A\frac{u^2}{1 + u} - Bu - uv\\
\dot v = - Cv + uv
\end{cases}
\end{equation}

Рассмотрим все возможные неподвижные точки этой системы. Для этого найдём решения системы:
\begin{equation*}
\begin{cases}
A\frac{u^2}{1 + u} - Bu - uv = 0\\
 - Cv + uv = 0
\end{cases}
\end{equation*}
Таким образом особыми точками будут:
\begin{equation*}
\begin{aligned}
&O(0, 0);\\
&K \left(\frac{B}{A-B}, 0\right), \quad \text{при } A > B;\\
&L\left(C, \frac{AC}{1 + C} - B\right), \quad \text{при } \frac{AC}{1 + C} \geq B
\end{aligned}
\end{equation*}
Условие для точки $L$ также можно записать в виде $\frac{B}{A - B} - C \leq 0$. Заметим, что при $A \leq B$ условие для существования точки $L$ также не выполняется ни при каких $C$.

Рассмотрим Якобиан системы (\ref{undist:syst}):
\begin{equation*}
J(u, v) = \begin{pmatrix} A \frac{2u + u^2}{(1 + u)^2} - B - v& \;  -u &\\ v &\; u- C &\end{pmatrix}
\end{equation*}
Исследуем особые точки на устойчивость.
\begin{enumerate}

\item Рассмотрим $O(0, 0)$. Якобиан в точке $O$ равен:
\begin{equation*}
J(0, 0) = \begin{pmatrix}  - B &\;  0 &\\ 0 &\; - C &\end{pmatrix}
\end{equation*}
Собственные значения $\lambda_1 = -B$, $\lambda_2 = -C$. Таким образом получаем, что $O$~---~узел и при этом асимптотически устойчивый.

\item Рассмотрим $K \left(\frac{B}{A-B}, 0\right)$. Якобиан в точке $O$ равен:
\begin{equation*}
J(\frac{B}{A-B}, 0) = \begin{pmatrix}  \frac{B (A -B)}{A} &\;  - \frac{B}{A - B} &\\ 0 &\; \frac{B}{A - B} - C &\end{pmatrix}
\end{equation*}
Собственные значения $\lambda_1 = \frac{B(A - B)}{A - B}$, $\lambda_2 = \frac{B}{A - B} - C$. При $\frac{B}{A - B} - C \neq 0$ получаем неустойчивую точку, в противном случае точки $K$ и $L$ совпадают, и мы получаем неустойчивый узел. При $\frac{B}{A - B} - C > 0$~---~это узел, при $\frac{B}{A - B} - C < 0$~---~это седло.

\item Рассмотрим $L \left(C, \frac{AC}{1 + C} - B\right)$.  Якобиан в точке $L$ равен:
\begin{equation*}
J\left(\frac{B}{A-B}, 0\right) = \begin{pmatrix}  \frac{AC}{(1 + C)^2} &\;  -C &\\ \frac{AC}{1 + C} - B &\; 0 &\end{pmatrix}
\end{equation*}
Характеристический многочлен матрицы имеет следующий вид:
\begin{equation*}
\lambda^2 - \frac{AC}{(1 + C)^2} \lambda + C \left(\frac{AC}{1 + C} - B \right) = 0
\end{equation*}
Обозначим $M = \frac{AC}{(1 + C)^2}$, $N = C \left(\frac{AC}{1 + C} - B \right)$. Тогда имеем корни:
\begin{equation*}
\lambda_{1,2} = \frac{M \pm \sqrt{M^2 - 4 N}}{2}
\end{equation*}
При этом M>0, и тогда
\begin{itemize}
\item если $M^2 - 4 N < 0$, то это неустойчивый фокус;
\item если $M^2 - 4 N = 0$, то это вырожденный узел;
\item если $M^2 - 4 N > 0$, то это неустойчивый узел.
\end{itemize}
На практике получается, что не существует положительных параметров $A$, $B$ и $C$, удовлетворяющих первым двум условиям, однако неравенство получается слишком громоздким, а потому рассматриваться не будет.
\end{enumerate}

Соберём все полученные результаты в таблицу:

\begin{tabular}{|c|c|c|c| }
\hline
\textbf{Условие} & \textbf{Точка $O$} & \textbf{Точка $K$} & \textbf{Точка $L$}  \\ \hline
$A \leq B$ & Устойчивый узел & Отсутствует & Отсутствует\\ \hline
$A > B$,  & Устойчивый узел & Неустойчивый узел & Отсутствует\\ 
$\frac{B}{A - B} - C > 0$ & & & \\ \hline
$A > B$,  & Устойчивый узел & Неизвестно & Неустойчивый узел\\ 
$\frac{B}{A - B} - C = 0$ & &  (совпадает с $L$) &  (совпадает с $K$) \\ \hline
$A > B$,  & Устойчивый узел & Неустойчивое седло & Неустойчивый узел\\ 
$\frac{B}{A - B} - C < 0$ & & & \\ \hline
\end{tabular}

Также рассмотрим изоклины системы (\ref{undist:syst}). В данном случае они будут задаваться следующими уравнениями:
\begin{equation*}
\begin{aligned}
&u = C;\\
&v = A \frac{u}{1 + u} - B.
\end{aligned}
\end{equation*}
При этом каждая изоклина разделяет фазовое пространство на части следующим образом:
\begin{itemize}

\item при $u < C$ $v$ убывает;

\item при $u > C$ $v$ возрастает;

\item при $v > A \frac{u}{1 + u} - B$ $u$ убывает;

\item при $v < A \frac{u}{1 + u} - B$ $u$ возрастает.

\end{itemize}
На картинках изоклины обозначены фиолетовым цветом

\clearpage

\section{Иллюстрации}
 \begin{figure}[h!]
\begin{center}
%\includegraphics[width=0.7\linewidth]{fig_1}
\caption{Фазовый портрет системы при $A=1$, $B=2$ и $C = 1$, соответствует 1 случаю из таблицы}
\label{fig_1}
\end{center}
\end{figure}

 \begin{figure}[h!]
\begin{center}
%\includegraphics[width=0.7\linewidth]{fig_2}
\caption{Фазовый портрет системы при $A=3$, $B=2$ и $C = 1$, соответствует 2 случаю из таблицы}
\label{fig_2}
\end{center}
\end{figure}

 \begin{figure}[h!]
\begin{center}
%\includegraphics[width=0.7\linewidth]{fig_3}
\caption{Фазовый портрет системы при $A=4$, $B=3$ и $C = 3$, соответствует 3 случаю из таблицы}
\label{fig_3}
\end{center}
\end{figure}

 \begin{figure}[h!]
\begin{center}
%\includegraphics[width=0.7\linewidth]{fig_4}
\caption{Фазовый портрет системы при $A=3$, $B=1$ и $C = 3$, соответствует 4 случаю из таблицы}
\label{fig_4}
\end{center}
\end{figure}

\clearpage

\section{Поиск волновых решений}

Найдём решения системы (\ref{trans:syst}) в следущем виде:
\begin{equation*}
\begin{aligned}
&u(t, x) = u(x + ct),\\
&v(t, x) = v(x + ct),
\end{aligned}
\end{equation*}
где $c > 0$~---~новый параметр. И обозначив $x + ct$ как новую переменную $z$, получим следующую систему:
\begin{equation*}
\begin{cases}
 c u'(z) = A\frac{u(z)^2}{1 + u(z)} - Bu(z) - u(z)v(z) + D u''(z)\\
c v'(z) = - Cv(z) + u(z)v(z) + v''(z)
\end{cases}
\end{equation*}

Рассмотрим частный случай, когда $D=0$. Он соответствует ситуации, когда все жертвы сосредоточены в одном месте, а хищники распределены повсюду. Также введём дополнительную  функцию $w(z) = v'(z)$, тогда систему можно переписать следующим образом:
\begin{equation}
\begin{cases}\label{dist:syst}
 u' = \frac{1}{c} \left( A\frac{u^2}{1 + u} - Bu - uv \right) \\
v' = w \\
w' = C v + cw - uv
\end{cases}
\end{equation}
Исследуем эту систему. Положения равновесия будут следующими:
\begin{equation*}
\begin{aligned}
&O(0, 0, 0),\\
&N\left(\frac{B}{A-B}, 0, 0\right), \quad \text{при } A>B,\\
&M \left(C, \frac{AC}{1 + C} - B, 0\right), \quad \text{при } \frac{AC}{1 + C} - B \geq 0.
\end{aligned}
\end{equation*}
Условие для точки $M$ также можно записать в виде $\frac{B}{A - B} - C \leq 0$. Также существование точки $M$ невозможно без существования точки $N$.

Рассмотрим Якобиан системы (\ref{dist:syst}):
\begin{equation*}
J(u, v, w) = \begin{pmatrix} - \frac{1}{c} \left(A \left(\frac{1}{(u+1)^2} - 1 \right) + B + v \right)& \;  -u & \; 0 &\\ 0 &\; 0 & \; 1 & \\ -v & \; C - u & \; c &\end{pmatrix}
\end{equation*}

Исследуем особые точки:
\begin{enumerate}

\item Рассмотрим точку $O(0, 0, 0)$. Якобиан в этой точке равен:
\begin{equation*}
J(u, v, w) = \begin{pmatrix} - \frac{B}{c} & \;  0 & \; 0 &\\ 0 &\; 0 & \; 1 & \\ 0 & \; C & \; c &\end{pmatrix}
\end{equation*}
Собственные значения $\lambda_1 = -\frac{B}{c}$, $\lambda_2 = \frac{1}{2} \left( c \pm \sqrt{4 C + c^2} \right)$. Эта точка является неустойчивой.

\item Рассмотрим точку $M \left(C, \frac{AC}{1 + C} - B, 0\right)$. Якобиан в этой точке равен:
\begin{equation*}
J\left(C, \frac{AC}{1 + C} - B, 0\right) = \begin{pmatrix} - \frac{AC}{(C+1)^2} & \;  -C & \; 0 &\\ 0 &\; 0 & \; 1 & \\ -\frac{AC}{1 + C} + B & \;0 & \; c &\end{pmatrix}
\end{equation*}
Также рассмотрим характеристический многочлен этой матрицы:
\begin{equation*}
f(\lambda) = \lambda^3 + \left(\frac{AC}{(1 + C)^2} - c\right) \lambda^2 - \frac{ACc}{(1+C)^2}\lambda - \frac{A C^2}{1+C} + BC
\end{equation*}
Найти собственные значения не представляется возможным, однако можно сделать некоторые выводы о них, рассмотрев характеристический многочлен. Для этого найдём его производную:
\begin{equation*}
f'(\lambda) = 3 \lambda^2 + 2 \left(\frac{AC}{(C+1)^2} - c\right)\lambda - \frac{ACc}{(C+1)^2}.
\end{equation*}
Её корни имеют следующий вид:
\begin{equation*}
\lambda_{1,2} = \frac{1}{6} \left(\pm\sqrt{\left(\frac{2AC}{(C+1)^2} - 2c\right)^2 + \frac{12 ACc}{(C+1)^2}} - \frac{2AC}{(C+1)^2} + 2c\right),
\end{equation*}
при этом один из этих корней отрицательный, а один~---~положительный. Подставим $0$ в $f'(\lambda)$, и получим отрицательное число, то есть в нуле она убывает, при этом если подставить $0$ в $f(\lambda)$, то мы получим снова отрицательное число (из условия существования точки). Тогда в силу того, что функция будет иметь вид как на рис. \ref{fig_5}, можно однозначно сказать, что хотя бы один из корней будет вещественным и положительным. А значит, точка будет неустойчивой. 

\item Рассмотрим точку $N\left(\frac{B}{A-B}, 0, 0\right)$. Якобиан в этой точке равен:
\begin{equation*}
J\left(\frac{B}{A-B}, 0, 0\right) = \begin{pmatrix} \frac{B(A-B)}{Ac} & \;  -\frac{B}{A-B} & \; 0 &\\ 0 &\; 0 & \; 1 & \\ 0 & \; C -\frac{B}{A-B}& \; c &\end{pmatrix}
\end{equation*}
Собственные значения $\lambda_1=\frac{B(A-B)}{Ac}$, $\lambda_{2,3} = \frac{\pm \sqrt{A-B} \sqrt{4AC + Ac^2 - 4BC - Bc^2 - 4B} + Ac - Bc}{2(A-B)}$. $\lambda_1>0$ при выполнении всех условий, а потому точка будет неустойчивой. Также $\lambda_2$ (с положительным корнем) также будет больше нуля при параметрах, удовлетворяющих условиям.

\end{enumerate}

Таким образом все три точки будут неустойчивыми. Волновых решений для системы существовать не будет, потому как при любых значениях параметров один из видов вымирает, а популяция другого устремляется в бесконечность. Изоклины для данной системы будут следующими:
\begin{equation*}
\begin{aligned}
&u = C + cw;\\
&v = \frac{1}{c} \left(\frac{Au}{1+u} - B\right);\\
&w = 0.
\end{aligned}
\end{equation*}

%\clearpage
\section{Иллюстрации}
 \begin{figure}[h!]
\begin{center}
%\includegraphics[width=0.7\linewidth]{fig_5}
\caption{Пример вида функции $f(\lambda)$.}
\label{fig_5}
\end{center}
\end{figure}

 \begin{figure}[h!]
\begin{center}
%\includegraphics[width=0.7\linewidth]{fig_6}
\caption{Фазовый портрет системы при $A=1$, $B=4$, $C = 1$, $c=4$, только точка $O$}
\label{fig_6}
\end{center}
\end{figure}

 \begin{figure}[h!]
\begin{center}
%\includegraphics[width=0.7\linewidth]{fig_7}
\caption{Фазовый портрет системы при $A=5$, $B=3$, $C = 1$, $c=4$, точки $O$ и $N$}
\label{fig_7}
\end{center}
\end{figure}

 \begin{figure}[h!]
\begin{center}
%\includegraphics[width=0.7\linewidth]{fig_8}
\caption{Фазовый портрет системы при $A=9$, $B=1$, $C = 1$, $c=4$, точки $O$, $M$ и $N$}
\label{fig_8}
\end{center}
\end{figure}
\end{document}